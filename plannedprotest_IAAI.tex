%File: formatting-instruction.tex
\documentclass[letterpaper]{article}
\usepackage{caption}
%\DeclareCaptionType{copyrightbox}
%\graphicspath{{./figures/}}
\usepackage{subcaption}
\usepackage{url}
\usepackage{graphicx}
\usepackage{epstopdf}
\usepackage{multirow}
\usepackage{amsmath, amsfonts, amssymb}
\graphicspath{{./figures/}}
\usepackage{color}
%package for proper unicode rendering
\usepackage[utf8]{inputenc}
\usepackage{url}
\usepackage{aaai}
\usepackage{times}
\usepackage{helvet}
\usepackage{courier}

\setlength{\titlebox}{2.8in}
\inputencoding{utf8}

\frenchspacing
\setlength{\pdfpagewidth}{8.5in}
\setlength{\pdfpageheight}{11in}
\pdfinfo{
/Title (Insert Your Title Here)
/Author (Put All Your Authors Here, Separated by Commas)}
\setcounter{secnumdepth}{0}  
 \begin{document}
\newcommand{\narenc}[1]{[{\color{red} Naren writes: \it #1}]}
\newcommand{\sathappanc}[1]{[{\color{blue} Sathappan writes: \it #1}]}
\newcommand{\then}{\Rightarrow}
\newcommand{\softor}{\operatornamewithlimits{\tilde{\vee}}}
\newcommand{\softand}{\operatornamewithlimits{\tilde{\wedge}}}
\newcommand{\softthen}{\operatornamewithlimits{\tilde{\then}}}
\newcommand{\softneg}{\operatornamewithlimits{\tilde{\neg}}}
\title{Forecasting Protests \\by Detecting Future Time Mentions \\in News and Social Media}
\author{
% 1st. author
Sathappan Muthiah\\
Virginia Tech\\
Arlington, VA 22203\\
sathap1@cs.vt.edu
% 2nd. author
\And
Bert Huang\\
       University of Maryland\\
       College Park, MD 20742\\
       bert@cs.umd.edu
%% 3rd. author
       \And
Jaime Arredondo\\
       University of California\\
       San Diego, CA 92093\\
       jarredon@ucsd.edu
\AND 
% use '\and' if you need 'another row' of author names
%% 4th. author
David Mares\\
       University of California\\
       San Diego, CA 92093\\
       dmares@ucsd.edu
%% 5th. author
       \And
Lise Getoor\\
       University of California\\
       Santa Cruz, CA 95064\\
       getoor@soe.ucsc.edu
%% 6th. author
       \And
Graham Katz\\
       CACI Inc.\\
       Lanham, MD 20706\\
       ekatz@caci.com
%\AND
       \And
%% 7th. author
Naren Ramakrishnan\\
       Virginia Tech\\
       Arlington, VA 22203\\
       naren@cs.vt.edu
}
\maketitle
\begin{abstract}
\begin{quote}
Identifying 'Calls for Protest' from Social Media

\end{quote}
\end{abstract}
\vspace{-1em}
\label{intro}
\section{Introduction}
%\begin{figure}
%    \includegraphics[width=0.5\textwidth, height=0.4\textwidth]{pp_example}
%    \caption{An example article describing plans for a future protest (Venezuela, June 11, 2014).}
%    \vspace{-2em}
%    \label{pp_example}
%\end{figure}
Civil unrest (protests, strikes, and ``occupy'' events) is a common happening in both democracies
and authoritarian regimes.
Although we typically associate civil unrest with disruptions and instability, for a social scientist
civil unrest reflects the democratic process by 
which citizenry communicate their views and preferences to those in authority. 
The advent of social
media has afforded citizenry new mechanisms for organization and mobilization, and online news sources
and social networking sites like Facebook and Twitter
can provide a window into civil unrest happenings in a particular country.

\subsubsection{Why study and forecast protests?}
Our region of interest is Latin America and protest
is an important topic of study here,
as many countries here are democracies struggling to consolidate themselves. The combination of weak channels of communication 
between citizen and government, and a citizenry that still has not grasped the desirability of elections as the means to affect politics 
means that public protest will be an especially attractive option. To 
illustrate the power of protest in Latin America we need only recall 
that between 1985 and 2011, 17 presidents resigned or were impeached under pressure from demonstrations, usually violent, in the streets. Protests have 
also resulted in the rollback of price increases for public services, such as during the ‘Brazilian Spring’ of June 2013.

Forecasting protests is an important capability in many domains.
For the tourism industry, forecasting protests can
support the issuance of travel warnings. For law enforcement,
forecasting protests can aid in preparedness. For the social scientist,
protests forecasts will provide insight into how citizens express themselves.
For the government, a protest forecasting system can help prioritize
citizen grievances. Finally, protests can have a debilitating effect on
multiple industries (esp. those that rely on worldwide supply chain management)
and thus a protest forecasting system can aid in planning and design
of alternative travel and shipping routes.

\subsubsection{Planned protests}
Our basic hypothesis is that
protests that are larger will be more disruptive and communicate support for its cause better than smaller protests. 
Mobilizing large numbers of people is more likely to occur if a protest is organized and the time and place announced in
advance. Because protest is costly and more likely to succeed if it is large, we should expect planned, rather than 
spontaneous, protests to be the norm. Indeed, in a sample of 288 events from our study selected for qualitative review of their antecedents,
for 225 we located communications regarding the upcoming occurrence of the event in media, and only 49 could be classified as 
spontaneous (we could not determine whether communications had or had not occurred in the remaining 14 cases).

\subsubsection{EMBERS}
We are an industry-university partnership charged with developing an
automatic protest forecasting system for 10 countries in
Latin America. Our system, called EMBERS, has been
deployed since Nov 2012 and has been generating forecasts (called
warnings or alerts) automatically, without a human-in-the-loop. These forecasts are emailed to
a third party (MITRE) for evaluation. Analysts at MITRE organize a reference
database of protests (called the Gold Standard Report,
or GSR) by surveying newspapers for reportings of protests, and
compare our warnings against the GSR to generate a scoring report (evaluation
criteria described later).

The full EMBERS system has been described elsewhere~\cite{emberskdd}, including
the overall system architecture, data sources used for analysis, and the
various forecasting models in EMBERS. EMBERS adopts a multi-model approach,
wherein different models are leveraged for their selective superiorities
to generate a fused set of alerts. Arguably, one of the
best performing models in EMBERS is the planned protest model that detects
ongoing organizational activity and generates warnings accordingly. This paper
is the first to present this model in detail, including the 
research issues involved, and how we addressed them in EMBERS.

Capturing mentions of protest planning and organization 
is not as easy as it might appear. First, articles of interest are written in
different languages (Spanish, Portugese, French, Dutch, and English). 
Second, multiple locations are often mentioned in a given article, leading
to (natural language) ambiguity about the intended location of the event.
Significant reasoning is required to discern the correct protest location.
Finally, dates are often described in relative terms, e.g., `Sunday' and 
thus these vague references need to be resolved into absolute temporal
information. 

Our detection approach 
combines shallow linguistic analysis to identify a corpus of relevant
documents (articles, tweets) which are then subject to targeted deep semantic analysis.
Despite its simplicity, we are able to
detect indicators of event planning with surprisingly high
accuracy. Our contributions are:

\iffalse
\begin{figure}
    \includegraphics[width=0.5\textwidth]{alertstructure}
    \vspace{-2em}
    \caption{An example warning (left) and GSR event (right).}
    \label{fig:alertstructure}
\end{figure}
\fi

\begin{enumerate}
\item We present a protest forecasting system that couples three key technical ideas:
key phrase learning to identify what to look for, probabilistic soft logic to reason about location occurrences in extracted results, and 
date normalization to resolve future tense mentions. We demonstrate how the integration of these ideas achieves objectives in precision,
recall, and quality (accuracy).
\item We illustrate the application of our system to 10 countries in Latin America, viz. Argentina, Brazil, Chile, Colombia, Ecuador, El Salvador, Mexico, Paraguay, Uruguay, and Venezuela. Our system predicts the {\it when} of the protest
as well as {\it where} of the protest (down to a city level granularity).
%See Fig.~\ref{fig:alertstructure} for an example.
We conduct ablation studies to identify the 
relative contributions of news media (news + blogs) versus social media (Twitter, Facebook) to identify future happenings of
civil unrest. Through these studies we illustrate the selective superiorities of different sources for specific countries.
\item Unlike many studies of retrospective forecasting of protests,
our system has been {\bf deployed and in operation for nearly two years.}
The end consumers of our alerts are analysts studying Latin America.
%To assess the forecasting prowess of our approach,
%we calculate the lead time from when the forecast is made to
%the actual event date, 
Our results demonstrate that we are able to 
capture significant societal unrest in the above countries with an average lead time of 4.08 days. 

\end{enumerate}



\vspace{-1em}
\section{Related Work}
%lGroup related work into three parts.
%
%Protest detection and forecasting. Include HRL, both detection and 
%forecasting. HRL has papers in ISI 2013. Add VT papers.
%
%Detecting planned events. Columbia work. Look at references in that
%paper to identify more. Event detection research (cite 10-15 papers).
%Oren Etzioni, his work.
%
%NEED ONE MORE SUBGROUP.
%
%
%\begin{itemize}
%    \item
%        \textbf{HRL's Twitter Planned Protest Paper: Detecting future social unrest in unprocessed Twitter}: 
%
%        {\em keywords}: 335 keywords identified by domain experts, used to filter twitter stream.
%
%        {\em Date}: Quoting "Our future dates is done."Our future date filter searches first for month names and abbreviations in Spanish and Portuguese and second for numbers less than 31 within three whitespace separated tokens from each other. Thus, an example matching date pattern would be ”10 de enero”.
%
%
%        {\em Probability}: Posterior probability of tweet being Civil Unrest related Given User-type
%
%        {\em Event Geocoding}: If no location in Text then location of retweeter is used based on HRL's Geocoding methodology.
%
%        {\em EventType and Population}: Duplicate Forecasts for same date/location are combined into one and then tweet history of every tweeter/retweeter in the forecast are searched for keywords pertaining to particular classes identified by domain expert and then most commonly occuring class is assigned.
%        
%        {\em Results Provided}: Counts of Warnings provided and results of Manual examination of warning provided
%
%    \item \textbf{HRL's Tumblr Based Paper- Civil Unrest Prediction: A Tumblr-Based Exploration}
%
%        {\em keywords}: 59 keywords by domain experts.
%
%        {\em location}: Text Based filter for pre-defined locations(1022 in number)
%
%        {\em date}: Similar to the Twitter paper
%
%        {\em Dataset}: Full tumblr firehose of public post from 2013-04 to 2013-11. Evaluations from 2013-06 to 2013-08. Results presented in terms of Number Events Detected, Precision, Leadtime
%
%    \item \textbf{Identifying Content for Planned Events Across Social Media Sites - Luis Gravano-Columbia University}:
%            Not exactly related to ours. It deals with identifying user contributed content for future planned events that are already known like music concert etc.
%
%\end{itemize}

Three categories of related work -- \emph{Event Detection, Extraction of Planned Events and  Event Forecasting} -- are briefly discussed here.

First, Event Detection/Extraction from textual News has been studied extensively in literature.A lot of them make use of document clustering techniques \cite{Allan:2002:TDT} \cite{Yang:1998:SRO}\cite{Gabrilovich:2004:NPP}to identify events either retrospectively or as the stories arrive.\cite{Chambers:2011:TIE},\cite{Banko07openinformation}, \cite{Riloff:2003:LEP} talk about extraction patterns/templates to extract information from Text. \cite{Ritter:2012} is one of the first papers to show it's possible to accurately extract a calendar of significant events from Twitter by training a tagger for recognizing event phrases.\cite{Becker:2012:ICP}, \cite{Becker_automaticidentification} deal with identifying content for already known planned events from Social Media.\cite{Sankaranarayanan:2009:TNT} captures tweet clusters of interest to identify late breaking News from twitter. In an altogether different application \cite{Sakaki:2010:EST} observes tweets to enable detection of occurences of Earth Quakes promptly.

Second, not much work is done regarding extraction of planned/future mentions of events from Social Media. Recorded Future\cite{recordedFuture} is an analytics company that performs real-time analysis of news and tweets to identify mentions of future events.\sathappanc{need to write more on how it is different from ours?}.In \cite{Becker:2012:ICP} and \cite{Becker_automaticidentification} content about known planned events is identified from Social Media.



Third, few work has been done in the area of Event Forecasting. \cite{Radinsky:2013:MWP}, is one of the first papers in this category. The paper learns event sequences from a corpora spanning over 22 years and then uses these sequences to forecast events like disease outbreaks, deaths and riots.It only predicts if an event of interest will happen in the future given the sequence of events seen but it does not predict when/where(city level) that event will happen. In \cite{DBLP:journals/corr/Kallus14}, the author makes use of data from Recorded Future to find if a  significant protest event will occur in the next three days or not usig a random forest classifier.The author only focuses on prediction of significant events and also the forecast is limited to the next three days.\cite{compton2013detecting} and \cite{xu2014civil} are two pieces of research  that are very close to our line of work. Both papers follow similar methodologies but are based on different datasets--Twitter and Tumblr.In \cite{compton2013detecting}, a list of 335 keywords identified by experts is used to filter twitter stream and the filtered twitter streams are then searched for the presence of future dates in a naive manner by first searching for month names and then for a number less than 31. Such an approach, will not be capable of finding relative mentions of future dates like "tomorrow", "next tuesday" etc. If any location is mentioned in the text of the tweet it is used as the event location else the location is determined based on a in-house geocoder\cite{hrlgeocoder}.\cite{xu2014civil} works on Tumblr and makes uses of a fewer set of keywords(59) to filter the Tumblr feed. The filtered feed is further refined by searching for mentions of around 1022 different location names.Finally, the documents are searched for a future date in the same way as in \cite{compton2013detecting}.


\vspace{-1em}
\section{Probabilistic Soft Logic}
%!TEX root = ../plannedprotest.tex

To extract the protest location from news articles, we use \emph{probabilistic soft logic} (PSL) \cite{broecheler:uai10;kimmig:probprog12} to build a model that performs robust, probabilistic inference given noisy signals. PSL takes a set of weighted, logic-like rules and converts them into a continuous probability distribution over the unknown truth values of logical facts. These truth values in PSL are relaxed into the $[0,1]$ interval. We use this mechanism to build a model that infers the semantic location of an article by weighing evidence coming from the Basis entity extractions and information in the World Gazatteer. 

The primary rules in the model encode the effect that Basis-extracted location strings that match to gazatteer aliases are indicators of the article's location, whether they be country, state, or city aliases. Each of these implications is conjuncted with an prior for ambiguous, overloaded aliases that is proportional to the population of the gazetteer location. For example, if the string ``Los Angeles'' appears in the article, it could refer to either Los Angeles, California, or Los \'{A}ngeles in Argentina or Chile. Given no other information, our model would infer a higher truth value for the article referring to Los Angeles, California, because it has a much higher population than the other options. 

The secondary rules, which are given half the weight of the primary rules, perform the same mapping of extracted strings to gazetteer aliases, but for extracted persons and organizations. Strings describing persons and organizations often include location clues (e.g., ``mayor of Buenos Aires''), but intuition suggests the correlation between the article's location and these clues may be lower than with location strings. 

Finally, the model includes rules and constraints to require consistency between the different levels of geolocation, making the model place higher probability on states with its city contained in its state, which is contained in its country. As a post-processing step, we enforce this consistency explicitly by using the inferred city and its enclosing state and country, but adding these rules into the model makes the probabilistic inference prefer consistent predictions, enabling it to combine evidence at all levels.
\label{section:PSL}

\vspace{-1em}
\section{Approach}

All News,Blogs and Tweets are first searched for the phrases learned. Then the filtered documents are searched for the presence of a reference to a Future Date. In case of News/blogs we search for the Presence of a reference to a Future Date only within the sentence where the phrase was found to reduce error. For tweets, we search the entire tweet for the reference of a future date.

Then, finally, a warning/alert is issued for those documents which contains a location information. In the case of tweets, we found that issuing an alert from just one tweet lead to a lot of wrong alerts. We thus, further filter the tweets by setting a threshold (set to 5) on the number of re-tweets of the tweet under consideration.

Each Individual step is discussed in detail below.

\begin{figure*}
\includegraphics[width=\textwidth]{pp_pipeline}
\caption{A diagram showing various steps of the Model}
\end{figure*}

\subsection{Learning of Phrases}
\sathappanc{Intro written for KDD paper}

{\em (Reference: Learning Extraction Patterns for Subjective expressions -- Ellen Riloff and Janyce Wiebe)}

Initially, a few seed phrases were obtained manually
with the help of subject matter experts. These phrases were parsed
using a dependency parser and the grammatical relationship between the
core subject word---{\em protest}, {\em manifestación}, {\em Huelga},
etc.---and any accompanying word -- {\em plan}, {\em call}, {\em anunciar} --- was extracted. To extend the initial set of phrases, a set of sentences/tweets containing a subject word and a
future time/date expression was collected and parsed.  This set of
sentences was used to expand the set of planned protest phrases by
extracting all keyword combinations that have the same grammatical
relation with respect to the core subject word. The final set of
planned protest phrases is then obtained after a manual revision of
the phrases obtained in the last step.

By this approach, we learned 122 phrases for News/blogs and 186 for tweets.


The learned phrases are then used to filter the incoming stream of Documents (news/blogs/twitter). The phrases matching is done by first splitting the incoming document into sentences and then looking for the presence of each individual word of a key-phrase (by lemma) separated by a pre-fixed maximum offset-distance (set to 3). This methodology greatly increases processing speed.

\begin{figure}
\caption{placeholder fig from an old ppt showing phrase learning}
\includegraphics[width=0.5\textwidth]{figures/phraseLearning}
\end{figure}



\subsection{classification}
For News/Blogs and Facebook, we make use of Text Based Naive Bayes Classifier to identify the event-type and population. Unigram and Bi-gram word features are used for training the classifier.

For Twitter, as we send alerts based on a single tweet, we chose the event-type and population based on prior likelihood for that location.



\vspace{-1em}
\section{Experiments}
\begin{itemize}
\item How Planned Protest model fared over the months?
Evaluation of Planned Protest Over the months

\item Contribution of Different Data Sources?
Evaluation of Each Individual Data Source,

\item Brazilian June Protests and Venezuelan February Protests 

\item Different Evaluation criteria
   1. Varying Time Constraint from 7 to 1
   2. Distance Based Scoring

\end{itemize}


\vspace{-1em}
\section{Development and Maintenance}
The core algorithms behind the
planned protest detector were implemented in Python. The PSL geocoder was
implemented in Java. The Basis Rosette Linguistic Platform is the key
external library utilized. The development process took 3 months (June
2012 to Aug 2012) and was
primarily led by the first author with contributions from the other authors.
After two months of testing (Sep 2012 and Oct 2012), the system was deployed
in Nov 2012. Since there is not an explicit training
phase, the system has required minimal re-engineering over time. Key changes
made to the system over time was to increase the sources used for data ingest
and supporting the inclusion of additional phrases. Agile software
engineering methods were used for project management.

\vspace{-1em}
\section{Discussion}
We have described an approach to forecasting protests by detecting mentions of future events in news and social
media. The two twin issues of i) resolving the date and ii) resolving the location have been addressed satisfactorily
to realize an effective protest forecasting system. As different forms of communication media gain usage, systems
like ours will be crucial to understanding the concerns of citizenry.

Our future work is aimed at three aspects. First, to address situations such as nationwide protests and systems of protests,
we must generalize our system from generating protests at a single article level to digesting groups of articles. This will
require more sophisticated reasoning using PSL programs. 
Second, we would like to generalize our approach that currently
does detection of overt plans for protest to not-so-explicitly stated expressions of discontent. 
Finally, we plan to consider other population-level events of interest than just civil unrest, e.g., domestic political crises,
and design detectors to recognize the imminence of such events.

\vspace{-0.8em}
{\small
\section*{Acknowledgments}
Supported by the Intelligence Advanced Research Projects Activity (IARPA) via
DoI/NBC contract number D12PC000337, the US Government is authorized to reproduce and distribute reprints of
this work for Governmental purposes notwithstanding any copyright annotation thereon.
Disclaimer: The views and conclusions contained herein are those of the authors and should not be interpreted as necessarily representing the official policies or endorsements, either expressed or implied, of IARPA, DoI/NBC, or the US Government.
}\vspace{-1em}
\small
\bibliographystyle{aaai}
\bibliography{references}
\end{document}
