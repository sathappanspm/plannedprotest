
All News,Blogs and Tweets are first searched for the phrases learned. Then the filtered documents are searched for the presence of a reference to a Future Date. In case of News/blogs we search for the Presence of a reference to a Future Date only within the sentence where the phrase was found to reduce error. For tweets, we search the entire tweet for the reference of a future date.

Then, finally, a warning/alert is issued for those documents which contains a location information. In the case of tweets, we found that issuing an alert from just one tweet lead to a lot of wrong alerts. We thus, further filter the tweets by setting a threshold (set to 5) on the number of re-tweets of the tweet under consideration.

Each Individual step is discussed in detail below.

\begin{figure*}
\includegraphics[width=\textwidth]{pp_pipeline}
\caption{A diagram showing various steps of the Model}
\end{figure*}

\subsection{Learning of Phrases}
\sathappanc{Intro written for KDD paper}

{\em (Reference: Learning Extraction Patterns for Subjective expressions -- Ellen Riloff and Janyce Wiebe)}

Initially, a few seed phrases were obtained manually
with the help of subject matter experts. These phrases were parsed
using a dependency parser and the grammatical relationship between the
core subject word---{\em protest}, {\em manifestación}, {\em Huelga},
etc.---and any accompanying word -- {\em plan}, {\em call}, {\em anunciar} --- was extracted. To extend the initial set of phrases, a set of sentences/tweets containing a subject word and a
future time/date expression was collected and parsed.  This set of
sentences was used to expand the set of planned protest phrases by
extracting all keyword combinations that have the same grammatical
relation with respect to the core subject word. The final set of
planned protest phrases is then obtained after a manual revision of
the phrases obtained in the last step.

By this approach, we learned 122 phrases for News/blogs and 186 for tweets.


The learned phrases are then used to filter the incoming stream of Documents (news/blogs/twitter). The phrases matching is done by first splitting the incoming document into sentences and then looking for the presence of each individual word of a key-phrase (by lemma) separated by a pre-fixed maximum offset-distance (set to 3). This methodology greatly increases processing speed.

\begin{figure}
\caption{placeholder fig from an old ppt showing phrase learning}
\includegraphics[width=0.5\textwidth]{figures/phraseLearning}
\end{figure}



\subsection{classification}
For News/Blogs and Facebook, we make use of Text Based Naive Bayes Classifier to identify the event-type and population. Unigram and Bi-gram word features are used for training the classifier.

For Twitter, as we send alerts based on a single tweet, we chose the event-type and population based on prior likelihood for that location.

