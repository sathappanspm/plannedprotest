%!TEX root = ../plannedprotest_IAAI.tex

\begin{figure*}
\includegraphics[width=\textwidth]{pipeline}
\caption{Schematic of the planned protest detector that ingests five
different types of data sources.}
\label{flowchart}
\end{figure*}
The general approach we adopt is to identify open-source documents
that appear to indicate civil unrest event planning, extract
relevant information from identified documents and use that as the
basis for a structured warning about the planned event. 
We ingest a wide array of textual documents, including RSS feeds (news and blogs),
mailing lists, contents of
URLs referenced in tweets, the contents of the tweets themselves,
and Facebook event pages.
All harvested documents are subjected to linguistic analysis; candidate
documents are identified using a list of (learned) phrases associated with
protest event planning; date and location information is extracted from the
text and reasoned about to generate a warning. Location information is normalized to conform to a standard (in our case, we use the World Gazetteer).
Each of these processing
steps (see Fig.~\ref{flowchart}) is outlined next.


\subsection{Linguistic Preprocessing}

Since our region of
interest is Latin America as well as the Middle East and North Africa, the collection of text harvested is
inherently multilingual, with Spanish, Portugese, Arabic, and English as the
dominating languages. Ingested documents are subjected to shallow
linguistic processing prior to analysis.
This initial processing involves identifying
the language of the document, distinguishing the words (tokenization),
normalizing words for inflection (lemmatization), and identifying
expressions referring to people, places, dates and other entities.
We use Basis Technology's Rosette Linguistics
Platform (RLP) suite of multilingual commercial tools
(\url{http://www.basistech.com/text-analytics/rosette/}) for this
processing.  The output of this linguistic preprocessing serves as input to
subsequent deeper analysis in which date expressions are normalized, sentences that appear to be describing protest planning are identified, 
and the geographic focus of the text computed.

\subsection{Date Normalization}
Date processing is particularly crucial to the identification and interpretation of statements about future events. We used the TIMEN~\cite{LlorensDGS12} date
normalization package to normalize date expressions 
in English, Spanish and Portuguese (extended the system to cover Portuguese) and the HeidelTime~\cite{strotgen2014time} to do the same for Arabic (extending the set of rules to handle Hijri dates).  Both systems rules were extended to improve coverage and accuracy on our document collection.

The systems makes use of meta-data such as the day of publication and
other information about the linguistic context to determine for each
date expression, what day (or week, month or year) it refers to. For
example in a tweet produced on June 10, 2014, the occurrence of the
term {\em Friday} used in a future-tense sentence {\em We'll get
  together on Friday} will be interpreted as June 13, 2014.  Each
expression identified as a date by the RLP preprocessor is normalized
in this way, with accuracy of just over 80% on our data set.

\subsection{Phrase filtering}
In order to identify relevant documents, input documents are filtered on a set of key phrases, i.e.,
the text of the document is searched for the presence of one or
more key phrases in a list of phrases which are indicative of an article's focus being
a planned civil unrest event.  
The list of key phrases indicating civil unrest planning was obtained
in a semi-automatic manner, as detailed below.
Articles which do match are processed further, those that do not are ignored.

\subsubsection{Phrase matching}
Our key phrase matching is highly general and linguistically
sophisticated.  The phrases in our list are general rules for
matching, rather than literal string sequences. Typically a phrase
specification comprises: two or more word lemmas, a language
specification, and a separation threshold. This indicates that words---potentially inflected forms---in 
a given sequence potentially separated by one or more other words, should be taken to be a
match. We determined that this kind of
multi-word key phrases was more accurate than simple keywords for
extracting events of interest from the data stream.

The presence of a keyphrase is checked by searching for the presence of
individual lemmas of the keyphrase within the same sentence separated
by at most a number of words that is fewer than the separation threshold.  
This method allows for linguistically sophisticated and flexible matching, so, for example,
the keyphrase {\bf [{\em plan protest}, 4, English]} would match the sentence
{\em The students are planning a couple big protests tomorrow} in an input document.

\subsubsection{Phrase list development}
\label{sec:phraselearning}
The set of key phrases was tailored (slightly) to the genre of the
input. In particular different phrases were used to identify relevant
news articles and blogs from those used to filter Tweets.  The lists
themselves were generated semi-automatically.

Initially, a few seed phrases were obtained manually
with the help of subject matter experts.
An analysis of news reports for planned protests in the print media helped create a small set of words that are frequently used in such articles.  We choose four nouns from are used predominantly to indicate a civil unrest
in the print media - {\em demonstration, march, protest} and
{\it strike} and translated them into Spanish and Portuguese (including
synonyms).  We then combined these with future-oriented verbs, e.g., {\em to organize}, {\em to prepare}, {\em to
plan}, and {\em to announce}, to form proto-planned-protest phrases. For twitter, shorter phrases were identified; these had
a more direct call for action, e.g., {\em marchar}, {\em manhã de mobilização}, {\em
  vamos protestar}, {\em huelga}.

To generalize this set of phrases, they were parsed
using a dependency parser~\cite{freeling} and the grammatical
relationship between the core nominal focus word (e.g., {\em protest}, 
{\em manifestación}, {\em huelga}) and any accompanying
word (e.g., {\em plan}, {\em call}, {\em anunciar}) was
extracted. These grammatical relations were used as extraction
patterns as in~\cite{riloff2003learning} to learn more phrases from a
corpus of sentences extracted from the data stream of interest
(either news/blogs or tweets). This corpus consisted of sentences that
contained any one of the nominal focus words and also had mentions of
a future date. The separation threshold for a phrase was also
learned, being set to the average number of words separating
the nominal focus and the accompanying word.

The set of learned phrases was then reviewed by a subject matter expert for quality contraol.  
Using this approach, we learned 112 phrases for news articles and blogs and 156 for tweets.  
This phrase learning process is illustrated in Fig.~\ref{fig:phraselearning}.

\begin{figure}
\includegraphics[width=0.5\textwidth]{figures/phraseLearning}
\caption{An example of phrase learning for detecting planned protests.}
\label{fig:phraselearning}
\end{figure}

\subsection{Geocoding}
\label{subsection:geocoding}
After linguistic preprocessing and suitable phrase filtering,
messages are geocoded with a
specification of the geographical focus of the text---specified as a
city, state, country triple---that indicates the locality that the
text is about. We make use of different geocoding methodologies
for Twitter messages, for Facebook Events pages, and for news articles and blogs.
These are described below.

\subsubsection{Twitter and Facebook}
For tweets, the geo-focus of the message is generated by a fairly
simple set of heuristics involving i)
the most reliable but least
available source, i.e., the geotag (latitude, longitude) of the tweet itself,
ii) Twitter {\it places} metadata, and iii) if the above are not
available, the text fields 
contained in
the user profile (location, description) as well as the tweet text
itself to find mentions of relevant locations.  Additional toponym disambiguiation heuristics are used to
identify the actual referent of the mention.
Similar methods are used to geocode event data extracted from 
Facebook event pages.  
%Since only Facebook events that have a venue are used and since the
% venue of a Facebook event generally specifies a latitude, longitude, and physical address information, 
%identifying the location is a fairly trivial task.  In cases where only latitude and longitude are given, 
%we apply reverse-geocoding mechanisms similar to those used for Twitter.

\subsubsection{News and Blogs}
For longer articles such as news articles, the geo-focus of the message is identified using much more complex methods
To extract the protest location from news articles, we use PSL to build probabilistic models that infer the intended
location of a protest by 
weighing evidence coming from the Basis entity extractions and
information in the World Gazetteer. 

The primary rules in the model encode the effect that Basis-extracted location strings that match to gazatteer 
aliases are indicators of the article's location, whether they be country, state, or city aliases. 
Each of these implications is conjuncted with an prior for ambiguous, overloaded aliases that is 
proportional to the population of the gazetteer location. For example, if the string ``Los Angeles'' appears in the article, 
it could refer to either Los Angeles, California, or Los \'{A}ngeles in Argentina or Chile. Given no other information,
our model would infer a higher truth value for the article referring to Los Angeles, California, because it 
has a much higher population than the other options. 
\begin{flalign*}
  \textsc{Entity}&(L, \textrm{location}) \softand \textsc{RefersTo}(L,\textrm{locID}) &\\
  &\rightarrow \textsc{PSLLocation}(\textrm{Article}, \textrm{locID}) &
\end{flalign*}

\begin{flalign*}
  \textsc{Entity}&(C, \textrm{location}) \softand \textsc{IsCountry}(C) &\\
  &\rightarrow \textsc{ArticleCountry}(\textrm{Article}, C) &
\end{flalign*}

\begin{flalign*}
  \textsc{Entity}&(S, \textrm{location}) \softand \textsc{IsState}(S)&\\
  &\rightarrow \textsc{ArticleState}(\textrm{Article}, S)&
\end{flalign*}
\noindent
(Note that the above are not deterministic rules; e.g., they do not use the logical conjunction $\wedge$ but rather the
Lukasiewicz t-norm based relaxation $\softand$. Further, these rules do not fire deterministically but are instead
simultaneously solved for satisfying assignments as described in Section~\ref{section:PSL}.)

The secondary rules, which are given half the weight of the primary rules, perform the same mapping of extracted strings 
to gazetteer aliases, but for extracted persons and organizations. Strings describing persons and 
organizations often include location clues (e.g., ``mayor of Buenos Aires''), but intuition suggests 
the correlation between the article's location and these clues may be lower than with location strings. 
\begin{flalign*}
  \textsc{Entity}&(O, \textrm{organization}) \softand \textsc{RefersTo}(O, \textrm{locID})&\\
                            &\rightarrow \textsc{PSLLocation}(\textrm{Article},\textrm{locID}) &
\end{flalign*}

\begin{flalign*}
  \textsc{Entity}&(O, \textrm{organization}) \softand \textsc{IsCountry}(O)&\\
  &\rightarrow \textsc{ArticleCountry}(\textrm{Article}, O)&
\end{flalign*}

\begin{flalign*}
  \textsc{Entity}&(O, \textrm{organization}) \softand \textsc{IsState}(O)&\\
  &\rightarrow \textsc{ArticleState}(\textrm{Article}, O) &
\end{flalign*}
Finally, the model includes rules and constraints to require consistency between the different levels of geolocation, 
making the model place higher probability on states with its city contained in its state, which is 
contained in its country. As a post-processing step, we enforce this consistency explicitly by using the 
inferred city and its enclosing state and country, but adding these rules into the model makes the 
probabilistic inference prefer consistent predictions, enabling it to combine evidence at all levels.
%{\color{red} As an example of how PSL aids in location identification, the example from Fig.~\ref{pp_example}
%is revisited in Fig.~\ref{fig:psl_example}. }
\begin{flalign*}
  \textsc{PSLLocation}&(\textrm{Article}, \textrm{locID}) \softand \textsc{Country}(\textrm{locID}, C)&\\
  &\rightarrow \textsc{ArticleCountry}(\textrm{Article}, C)&
\end{flalign*}

\begin{flalign*}
  \textsc{PSLLocation}&(\textrm{Article}, \textrm{locID}) \softand \textsc{Admin1}(\textrm{locID}, S)&\\
    &\rightarrow \textsc{ArticleState}(\textrm{Article}, S)&
\end{flalign*}
\begin{figure*}
\begin{tikzpicture}[
squaredred/.style={rectangle, draw=black!60, fill=red!5, very thick, minimum size=5mm},
squaredblue/.style={rectangle, draw=black!60, fill=blue!5, very thick,minimum size=5mm},
squaredgreen/.style={rectangle, draw=black!60, fill=green!5, very thick, minimum size=5mm},
]
%Nodes
\node[squaredred] (articlepic)
{\includegraphics[width=0.4\columnwidth,height=0.3\textheight]{figures/article}};

\node[squaredblue](geodist) [right=of articlepic, align=center, scale=0.75]
{\{"Admin1":"Caracas",\\
"city": "Caracas",\\
"country": "Caracas",\\
"confidence": 0.4218\}
\\
\\
\{"Admin1": "Miranda",\\
"city": "Baruta",\\
"country": "Venezuela",\\
"confidence": 0.2639\},
\\
\\
\{"Admin1": "Miranda",\\
"city": "Chacao",\\
"country": "Venezuela",\\
"confidence": 0.2639\},
\\
\\
\{"Admin1": "Ciego de Avila",\\
"city": "Venezuela",\\
"country": "Cuba",\\
"confidence": "0.0511"\},
\\
\\
\{"Admin1": "Cundinamarca",\\
"city": "El Rosal",\\
"country": "Colombia",\\
"confidence": 0.0012\},
};


\node[squaredgreen] (finalgeo) [right=of geodist, align=left] {Admin1: Caracas,\\
                                                 City: Caracas,\\
                                                 Country: Venezuela\\
                                                 Confidence: 0.4218};
%Lines
\draw[very thick, ->] (articlepic.east) -- (geodist.west);
\draw[very thick, ->] (geodist.east) -- (finalgeo.west);

\end{tikzpicture}
\caption{An example of location inference using PSL. Red circles
   denote named entities identified as locations and blue denotes other
   types of entities. The article describes students planning a march on
   Sunday.  It identifies multiple locations, e.g., Chacao, El Roso, and
   the Francisco Fajardo highway where protests have been happening.
   There is also a reference to a quote by the mayor of Baruto.
   Mentions of such multiple locations are resolved using our PSL
   program to the intended location, here Caracas.}
\label{fig:psl_example}

\end{figure*}


As an example of how PSL aids in location identification,
the protest from Fig.~\ref{pp_example} is revisited in
Fig.~\ref{fig:psl_example}, which illustrates
the evidence that the PSL model gathers from the news article and the
inferred locations.

