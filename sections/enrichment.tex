
\begin{table}
\caption{EMBERS system statistics}
 \centering
 \begin{tabular}{|l|l|l|l|l|}
 \hline
 Archived data     & 12.4 TB                  \\ \hline
 Archive size & ca. 3 billion messages   \\ \hline
 Data throughput   & 200-2000 messages/sec  \\ \hline
 Daily ingest & 15 GB \\ \hline
 System memory & 50 GB \\ \hline
 System core & 16 vCPUs \\ \hline
 System output & ca. 40 warnings/day \\ \hline
\end{tabular}
\label{tab:stats}
\end{table}


\subsection{Enrichment}

Messages with textual content (tweets, newsfeeds, blog postings, etc.)
are subjected to shallow linguistic processing prior to analysis. Note that
most of our content involves languages from the Latin American region, esp.
Spanish, Portuguese, but also French (and of course, English). Applying
BASIS technologies' Rosette Language Processing (RLP) tools, the language of
the text is identified, the natural language content is tokenized and
lemmatized and the named entities identified and classified. Date
expressions are normalized and deindexed (using the
TIMEN \cite{LlorensDGS12} package).  Finally, messages are geocoded with a specification
of the location (city, state, country), being talked about in the message.
An example of this enrichment processing can be seen in
Fig.~\ref{fig:enrichment}.


%\begin{itemize}
%\item \textbf{Basis Enrichment}: We make use of the Basis Rosette Language Processing tools to identify text language (spanish, portuguese etc.,), tokenize, lemmatize and identify named entities as well as classify them into Locations, Person and Organization.
%
%\item \textbf{TIMEN}: Date expressions are normalized and de-indexed using the TIMEN package -- {\em TIMEN: An Open Temporal Expression Normalisation Resource}.
%\end{itemize}
%

