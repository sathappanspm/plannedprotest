We make use of different geocoding methodologies for geo-coding news/blogs and twitter.
\subsubsection{News/Blogs}
Most news articles and blog posts mention multiple locations, e.g.,
the location of reporting, the location of the incident, and locations corresponding
to the hometown of the newspaper. We developed a probabilistic reasoning
engine using probabilistic soft logic (PSL)
to infer the most likely city, state and country which is the main geographic focus the article.The PSL geocoder combines various types of evidence, such as named entities
such as locations, persons, and organizations identified by RLP, as
well as common names and aliases and populations of known
locations. These diverse types of evidence are used in weighted rules
that prioritize their influence on the PSL model's location
prediction. For example, extracted location tokens are strong
indicators of the content location of an article, while organization
and person names containing location names are weaker but still
informative signals; the rules corresponding to these evidence types
are weighted accordingly.

The methodology is similar to {\em Web-a-where: Geo-Tagging Web Content}.
 
\subsubsection{Twitter}
The Twitter geocoding is achieved by first
considering the most reliable but least available source,
viz. geotags, which give us exact geographic locations that can be
reverse geocoded into place names.  Second, we consider Twitter places
and use place names present in these fields to geocode the place names
into geographical coordinates.  Finally, we consider the text fields
contained in the user profile (location, description) as well as the
tweet text itself to find mentions of relevant locations which can
then be geocoded into geographical coordinates.

\subsubsection{Facebook}
We make use of only the facebook event data for our experiments. Almost, all of the Facebook event pages contain information about the venue of the event which includes latitude, longitude, country, state, city, street etc. Under cases where only latitude and longitude is given we do a reverse-geocoding by a KD-Tree lookup from the World Gazetteer to get the country, state, city information. A very few event pages do not have sufficient location info, we ignore such pages that do not even contain country information.
