Protest is a means by which citizens communicate their views and preferences to those in authority. Representative government is based upon communication between governed and governors; ideally, the channels for communication are open, transparent, credible and efficient. Governments, nevertheless, find it difficult to know on any one issue and at any one time how their constituencies value the available options. Elections are retrospective indicators and rarely issue specific; polling taps into sentiment, but is not a good indicator of priorities or strength of feeling because of the low cost associated with responding. Events, on the other hand, indicate a willingness to bear some costs (organization, mobilization, identification) in support of an issue and thus reveal not only preferences but provide some indication of priorities.
 
Protest is especially important in democracies that are struggling to consolidate themselves, such as those in Latin America. The combination of weak channels of communication between citizen and government, and a citizenry that still has not grasped the desirability of elections as the means to affect politics means that public protest will be an especially attractive option. To illustrate the power of protest in Latin America we need only recall that between 1985 and 2011 17 Presidents resigned or were impeached under pressure from demonstrations, usually violent, in the streets. Protests have also resulted in the rollback of prices increases for public services, such as in Brazil in June 2013.
 
We can hypothesize the protests that are larger will be more disruptive and communicate support for its cause better than smaller protests. Mobilizing large numbers of people is more likely to occur if a protest is organized and the time and place publicized. Because protest is costly and more likely to succeed if it is large we should expect planned, rather than spontaneous, protests to be the norm. Indeed, in a sample of 288 events from our study selected for qualitative review of their antecedents, for 225 we located communications regarding the upcoming occurrence of the event and only 49 were classified as ‘spontaneous’ (we could not determine whether communications had or had not occurred in 14 cases).
