\sathappanc{write-up from KDD paper}
Many civil unrest events are planned and organized through calls-for-action
by opinion and community leaders who galvanize support for their case.
The planned protest model aims
at detecting such civil unrest events from
traditional media (e.g., news pages, mailing
lists, blogs) and from social media (e.g., Twitter, Facebook).
The model filters the input streams by matching to a custom
multi-lingual lexicon of expressions such as {\em preparación huelga},
{\em llamó a acudir a dicha movilización} or {\em plan to strike}
which are likely to indicate a planned unrest event.  The phrase
matching is done in flexible manner making use of the lemmatized,
tokenized output of the BASIS enrichment module, to allow for variation and
approximations in the matching.  Messages that match are then screened
for the mention of a future time/date occurring in the same sentence as the
phrase. The event type and population are forecast using a multinomial naive
Bayes classifier. Location information is determined using the enrichment geocoders.
%\narenc{Give an example of a news article, highlighted boxes around phrases,
%etc. and then show a warning record generated from it.}
%\grahamc{Can we reuse the figure from enrichment (Fig 5) as a starting point for this?}
The phrase dictionary is thus a crucial aspect of the planned protest
model and was populated in a semi-automatic manner using
both expert knowledge and a simple bootstrapping methodology.

\iffalse
Initially, a few seed phrases were obtained manually
with the help of subject matter experts. These phrases were parsed
using a dependency parser and the grammatical relationship between the
core subject word---{\em protest}, {\em manifestación}, {\em Huelga},
etc.---and any accompanying word was extracted. To extend the initial
set of phrases, a set of sentences containing a subject word and a
future time/date expression was collected and parsed.  This set of
sentences was used to expand the set of planned protest phrases by
extracting all keyword combinations that have the same grammatical
relation with respect to the core subject word. The final set of
planned protest phrases is then obtained after a manual revision of
the phrases obtained in the last step. \narenc{It would be good to give
an example of this phrase growing.}
\fi
The planned protest model reads three kinds of input messages:
standard natural language text (RSS news and blog feeds, as well
as the content of web pages mentioned in tweets), microblogging text
(Twitter), and Facebook Events pages.
The RSS feeds and web pages are processed as discussed above. For
tweets, in addition to the above processing, we require that the tweet
under consideration be retweeted a minimum number of
times, to avoid erroneous alerts. (This value is set to 20
in our system.) For Facebook, we use their public API to search
for event pages containing the word protest or its synonyms.
Most such Facebook event pages already
provide significant information such as the
planned date of protest, location (sometimes with resolution up to
street level), and population/category of
people involved.

