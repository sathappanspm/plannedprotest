In this section, we briefly describe probabilistic soft logic
(PSL)~\cite{kimmig2012short}, a key component of our geocoding strategy.
PSL is a framework for collective probabilistic reasoning on relational
domains.  PSL represents the domain of interest as logical atoms.  It
uses first order logic rules to capture the dependency structure of the
domain, based on which it builds a joint probabilistic model over all
atoms.  Instead of hard truth values of $0$ (false) and $1$ (true), PSL
uses soft truth values relaxing the truth values to the interval
$[0,1]$.  The logical connectives are adapted accordingly.

User defined \emph{predicates} are used to encode the relationships and
attributes and \emph{rules} capture the  dependencies and constraints.
The rules can also be labeled with non-negative weights which are used
during the inference process.  The set of predicates and weighted rules
thus make up a PSL program where known truth values of ground atoms
 are derived from observed data and unknown truth values for the remaining
atoms are learned using the PSL inference.

\begin{exmp}
We will follow a running example throughout this section. In our
geocoding subtask, we create a PSL program that reasons about the
predicate \textsc{RefersTo}, which maps a text string to real-world
locations. For example, $\textsc{RefersTo}(\textrm{``DC''},
\textrm{washingtonDC})$ evaluates to true if the knowledge base believes
that  the article refers to Washington D.C.~as ``DC.'' This predicate gets
used in rules that define dependencies between predicates, such as the
rule
\begin{flalign*}
  \centering
  \textsc{Entity}&(L, \textrm{location}) \wedge \textsc{RefersTo}(L,\textrm{locID}) &\\
  &\rightarrow \textsc{PSLLocation}(\textrm{Article}, \textrm{locID}), &
\end{flalign*}
which states that an entity extracted from an article text that matches
a known \textsc{RefersTo} mapping implies that the PSL program's
predicted location will follow that mapping. Some of these logical atoms
will be known as parts of a knowledge base, while others will be unknown
and will be inferred by PSL.
\end{exmp}

Given a set of atoms 
$\ell = \{\ell_1,\ldots,\ell_n\}$,
an interpretation defined as 
$I : \ell \rightarrow [0,1]^n$
is a mapping from atoms to soft truth values.  PSL defines a probability
distribution over all such interpretations such that those that satisfy
more ground rules are more probable.
\emph{Lukasiewicz t-norm} and its corresponding co-norm are used for
defining relaxations of the logical AND and OR respectively to determine
the degree to which a ground rule is satisfied.  Given an interpretation
$\mathit{I}$, PSL defines the formulas for the relaxation of the logical
conjunction ($\wedge$), disjunction ($\vee$), and negation ($\neg$) as
follows:
\begin{align*}
\ell_1 \wedge \ell_2 &= \max\{0, I(\ell_1) + I(\ell_2) - 1\},\\
\ell_1 \vee \ell_2 &= \min\{I(\ell_1) + I(\ell_2), 1\},\\
\neg l_1 &= 1 - I(\ell_1),
\end{align*}  

The interpretation $\mathit{I}$ determines whether the rules are
satisfied. A rule $\mathit{r} \equiv \mathit{r_{body}} \rightarrow
\mathit{r_{head}} $  is satisfied if and only if the truth value of head
is at least that of the body. Otherwise, PSL uses a \emph{distance to
satisfaction}, which measures the degree to which this condition is
violated
\begin{center} 
 $\mathit{d_r}(\mathit{I}) =$ max\{0,$\mathit{I(r_{body})} - \mathit{I(r_{head})}$\}.
 \end{center}
 
\begin{exmp}
Continuing the previous example, if we have the rule
\begin{flalign*}
  \centering
  \textsc{Entity}&(\textrm{``Washington''}, \textrm{location}) ~ \wedge\\
  & \textsc{RefersTo}(\textrm{``Washington''}, \textrm{washingtonDC}) &\\
  &~~~\rightarrow \textsc{PSLLocation}(\textrm{Article},
  \textrm{washingtonDC}), &
\end{flalign*}
and the truth values of the known atoms are 
\[
\mbox{\textsc{Entity} (``Washington'', location): 1.0} 
\]
and
\[
\mbox{\textsc{RefersTo}(\textrm{``Washington''}, \textrm{washingtonDC}): 0.6},
\]
then following the relaxation of logical AND, the truth value of the
antecedent is $\max\{0, 1.0 + 0.6 - 1\} = 0.6$. Moreover, if the truth
value of the atom to be inferred, \textsc{PSLLocation}(\textrm{Article},
\textrm{washingtonDC}), is 0.1, then the distance to satisfaction of
this rule is $\max\{0, 0.6 - 0.1\} = 0.5$. On the other hand, if the
truth value of the head is 0.7, then the distance to satisfaction is
$\max\{0, 0.6 - 0.7\} = 0$, meaning the rule is satisfied.
\end{exmp}

PSL then induces a probability distribution over possible
interpretations $\mathit{I}$ over the given set of ground atoms
$\mathit{l} $ in the domain.  If $\mathit{R}$ is the set of all ground
rules that are instances of a rule from the system and uses only the
atoms in  $\mathit{I}$ then, the probability density function
$\mathit{f}$ over $\mathit{I}$ is defined as
\begin{equation}
\label{eq:contimn1}
    f (I) = \frac{1}{Z} \exp \left(-\sum_{r\in R} \lambda_r (d_r(I))^p \right)
\end{equation}
\begin{equation}
\label{eq:contimn2}
	Z = \int_{I} \exp \left( -\sum_{r\in R} \lambda_r (d_r(I))^p \right)
\end{equation}

where $\lambda_r$ is the weight of the rule $r$, $Z$ is a normalization
constant, and $p \in {1, 2}$ provides a choice between linear or
quadratic loss functions, which produce different modeling behavior. PSL
further allows inclusion of linear equality and inequality constraints,
which enable modeling of functional constraints on the domains and
ranges of predicates. 

The probability distribution in Equation (1) is an example of a
\emph{hinge-loss Markov random field} \cite{bach:uai13}, which has a
form of energy function that makes inference of the most probable
explanation an efficient convex optimization. The expressiveness of PSL
and the efficiency of inference in its models allows us to encode
dependencies between various aspects of geolocation that are jointly
inferred.

\begin{exmp}
The joint probability distribution enables PSL to reason about
conflicting evidence, for example if we additionally have the atom
\mbox{\textsc{RefersTo}(``Washington'', Washington State)} in our
knowledge base with truth value 0.2, we would have two conflicting
\textsc{PSLLocation} implications.  The probabilistic, weighted rules,
as well as the soft truth values of known atoms control how much PSL
considers each piece of evidence, and additional corroborating or
conflicting evidence would also be incorporated into the final joint
inference.
\end{exmp}
