Six categories of related work 
%-- \emph{Event Detection, Temporal Information Extraction, Future Retrieval, Extraction of Planned Events and Event Forecasting} -- 
are briefly discussed here.

{\bf Event detection via text extractions}
is an extensively studied topic in the literature. Document clustering techniques are used 
in, e.g.,~\cite{Allan:2002:TDT, Gabrilovich:2004:NPP, Yang:1998:SRO}
to identify events retrospectively or as the stories arrive.
Works like~\cite{Chambers:2011:TIE, Banko07openinformation, riloff2003learning} focus on
extraction patterns (templates) to extract information from text. Ritter et al.~\cite{Ritter:2012} shows that
it is possible to accurately extract a calendar of significant events from Twitter by training a tagger for recognizing event phrases.
\iffalse 
Sankaranarayanan et al.\cite{Sankaranarayanan:2009:TNT} captures tweet clusters of interest to identify late breaking News from twitter 
\fi
Highly specialized applications
also exist; e.g., Sakaki et al.~\cite{Sakaki:2010:EST} mine tweets to enable prompt detection of occurences of earthquakes.

{\bf Temporal information extraction} is another well studied topic.
The TempEval challenge~\cite{tempeval} led to a significant amount of
algorithmic development for temporal NLP.
For instance, a specification lanuage
for temporal and event expressions in natural language text is described in~\cite{timeml}.
Refs.~\cite{LlorensDGS12} and \cite{tempex} provide methods to resolve temporal expressions in text (our own
work here uses the TIMEN package~\cite{LlorensDGS12}).

{\bf Future retrieval} is an emerging topic with 
Baeza-Yates~\cite{baeza2005searching} providing one of the earliest discussions
of this topic; here future temporal information in text is found and used to retrieve content from search queries that 
combine both text and time with a simple ranking scheme. Kawai et al.~\cite{Kawai:2010:CSE} present a search engine (ChronoSeeker) for searching 
future and past events.
They make use of an SVM classifier to disambiguate between the various temporal expressions in a document.
Dias et al.~\cite{dias2011future} classify web snippets into three classes depending on if a future date can/cannot be predicted 
from the snippet or if it is a rumor.

{\bf Extraction of planned or future mentions of events from social media} is a very popular topic in social media
analytics.
The company 
RecordedFuture~\cite{recordedFuture} \sathappanc{TODO: write how it differs from our application} real-time analysis of news and tweets to identify mentions of events along with associated times. Anectodally it is estimated that approximately (only) 5--7\% of events extracted 
by RecordedFuture are about the future.
Tops et al.~\cite{tops2013predicting} aim to classify a tweet talking about an event into discrete time segments and thereby predict the 
`time to event'.
Bosch et al.~\cite{bosch2013estm} use regression techniques to identify the time to an event referred to by a tweet.
Jatowt et al.~\cite{Jatowt:2011:ECE} provide a collective image of the future associated with an entity summarizing all future related information available.
Becker et al.~\cite{Becker:2012:ICP} identify content about known planned events across social media.
%\cite{baeza2005searching},\cite{dias2011future},\cite{tops2013predicting},\cite{Jatowt:2011:ECE},\cite{Kawai:2010:CSE} talk about future retrieval or extraction of mentions of Future events.

{\bf Event forecasting} is a burgeoning area. 
Radinsky and Horvitz~\cite{Radinsky:2013:MWP} find event sequences from a corpora and then use these sequences to determine if 
an event of interest (e.g., a disease outbreak, or a riot)
will occur sometime in the future.
\iffalse They only predict if an event of interest will happen in the future given the sequence of events seen but do not predict when/where(city level resolution) that event will happen \fi
Kallus~\cite{nathankallus} makes use of data from RecordedFuture~\cite{recordedFuture} to find if a  significant protest event will occur in 
the subsequent three days using a random forest classifier.
The author only focuses on prediction of significant events (suitably defined) and
the forecast is limited to the next three days.

Finally, two other publications---\cite{compton2013detecting} and~\cite{xu2014civil}---align very
closely to our own work as their emphasis is on {\bf protest forecasting}.
Both works are aimed at forecasting protests
but emphasize different data sources and different methodologies. For instance, the work in~\cite{compton2013detecting} filters the Twitter stream for
keywords of interest and searches for future date mentions in only absolute terms, i.e., explicit mentions of a month name and a number (date)
less than 31. 
Such an approach will not be capable of extracting the more
common way in which future dates are references, e.g., phrases like
``tomorrow,'' ``next tuesday.'' 
%Any location mentioned in the tweet text is used as the event location. If there are no location mentions then the location is determined based 
on \cite{hrlgeocoder}. The work in~\cite{xu2014civil} by the same group of authors uses the Tumblr feed with a smaller set of keywords but
again is restricted to the use of absolute time identifiers.

\begin{table*}
    \centering
    \caption{comparison of our approach with other Future Retrieval Techniques}
    \begin{tabular}{l c c c c c }
        \hline
        & Relative date extraction & Domain Specific & Multi-Source & Geo-Coding& \\
        \hline
        ML Based ~\cite{Kawai:2010:CSE, bosch2013estm, Jatowt:2011:ECE, tops2013predicting}&\checkmark & & &\\
        Pattern Based ~\cite{xu2014civil,compton2013detecting} & &\checkmark& & \checkmark\\
        our method &\checkmark &\checkmark &\checkmark&\checkmark\\ 
\end{tabular}
\end{table*}
