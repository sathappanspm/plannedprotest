Three categories of related work -- \emph{Event Detection, Extraction of Planned Events and  Event Forecasting} -- are briefly discussed here.

First, Event Detection/Extraction from textual News has been studied extensively in literature. \cite{Allan:2002:TDT} \cite{Yang:1998:SRO}\cite{Gabrilovich:2004:NPP} make use of document clustering techniques to identify events retrospectively or as the stories arrive.\cite{Chambers:2011:TIE},\cite{Banko07openinformation}, \cite{riloff2003learning} talk about extraction patterns/templates to extract information from text. \cite{Ritter:2012} shows it's possible to accurately extract a calendar of significant events from Twitter by training a tagger for recognizing event phrases.\cite{Sankaranarayanan:2009:TNT} captures tweet clusters of interest to identify late breaking News from twitter.In an altogether different application \cite{Sakaki:2010:EST} observes tweets to enable detection of occurences of Earth Quakes promptly.

Second, some work has been done regarding extraction of planned/future mentions of events from Social Media. RecordedFuture\cite{recordedFuture} \sathappanc{TODO: write how it differs from our application} is an analytics company that performs real-time analysis of news and tweets to identify mentions of future events.\cite{tops2013predicting} and \cite{bosch2013estm} use classification and regression techniques to identify the time to an event referred to by a tweet.In \cite{tops2013predicting} a tweet was classified into one of the several identified equal length time bins. \cite{Jatowt:2011:ECE} tries to provide a collective image of the future associated with an entity summarizing all future related information available.In \cite{Becker:2012:ICP} and \cite{Becker_automaticidentification} content about known planned events is identified from Social Media. \iffalse \sathappanc{we do it on-line and focus mainly on planned protest and do it in multiple sources} \fi
%\cite{baeza2005searching},\cite{dias2011future},\cite{tops2013predicting},\cite{Jatowt:2011:ECE},\cite{Kawai:2010:CSE} talk about future retrieval or extraction of mentions of Future events.

Temporal information extraction has been well studied in literature.The TempEval challenge\cite{tempeval} led to a lot of development with regards to temporal NLP.\cite{timeml} provides a specification language for temporal and event expressions in natural language text.
\cite{LlorensDGS12} and \cite{tempex} provides methods to resolve temporal expressions in text. We make use of TIMEN \cite{LlorensDGS12} for the work described here.

Several work has also been done on Future Retrieval(FR).\cite{baeza2005searching} is one of the earliest works regarding FR. It seeks for future temporal information in text and retrieve content from search queries that combine both text and time with a simple ranking scheme. \cite{Kawai:2010:CSE} presents a search engine ChronoSeeker for searching future and past events.It makes use of an SVM Classifier to disambiguate between the various temporal expressions in a document.\cite{dias2011future} try to classify web-snippets into 3 classes depending on if a future date can/cannot be predicted or if it is a rumor.

Third, few work has been done in the area of Event Forecasting. In \cite{Radinsky:2013:MWP} the authors learn event sequences from a corpora spanning over 22 years and then use these sequences to say if an event of interest (disease outbreaks, deaths and riots etc) will occur sometime in the future.\iffalse They only predict if an event of interest will happen in the future given the sequence of events seen but do not predict when/where(city level resolution) that event will happen \fi. In \cite{nathankallus}, the author makes use of data from RecordedFuture\cite{recordedFuture} to find if a  significant protest event will occur in the subsequent three days using a random forest classifier.The author only focuses on prediction of significant events and also the forecast is limited to the next three days.\cite{compton2013detecting} and \cite{xu2014civil} are two pieces of research  that are very close to our line of work. Both papers follow similar methodologies but are based on different datasets--Twitter and Tumblr.In \cite{compton2013detecting}, a list of 335 keywords identified by experts is used to filter twitter stream and the filtered twitter streams are then searched for the presence of future dates in a naive manner by first searching for month names and then for a number less than 31. Such an approach, will not be capable of finding relative mentions of future dates like "tomorrow", "next tuesday" etc. Any location mentioned in the tweet text is used as the event location. If there are no location mentions then the location is determined based on \cite{hrlgeocoder}.\cite{xu2014civil} works on Tumblr and makes uses of a fewer set of keywords(59) to filter the Tumblr feed. The filtered feed is further refined by searching for mentions of around 1022 different location names.Finally, the documents are searched for a future date in the same way as in \cite{compton2013detecting}.

\begin{table*}
    \centering
    \caption{comparison of our approach with other Future Retrieval Techniques}
    \begin{tabular}{l c c c c c }
        \hline
        & Relative date extraction & Domain Specific & Multi-Source & Geo-Coding& \\
        \hline
        ML Based ~\cite{Kawai:2010:CSE, bosch2013estm, Jatowt:2011:ECE, tops2013predicting}&\checkmark & & &\\
        Pattern Based ~\cite{xu2014civil,compton2013detecting} & &\checkmark& & \checkmark\\
        our method &\checkmark &\checkmark &\checkmark&\checkmark\\ 
\end{tabular}
\end{table*}
