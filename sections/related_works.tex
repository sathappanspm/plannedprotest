%lGroup related work into three parts.
%
%Protest detection and forecasting. Include HRL, both detection and 
%forecasting. HRL has papers in ISI 2013. Add VT papers.
%
%Detecting planned events. Columbia work. Look at references in that
%paper to identify more. Event detection research (cite 10-15 papers).
%Oren Etzioni, his work.
%
%NEED ONE MORE SUBGROUP.
%
%
%\begin{itemize}
%    \item
%        \textbf{HRL's Twitter Planned Protest Paper: Detecting future social unrest in unprocessed Twitter}: 
%
%        {\em keywords}: 335 keywords identified by domain experts, used to filter twitter stream.
%
%        {\em Date}: Quoting "Our future dates is done."Our future date filter searches first for month names and abbreviations in Spanish and Portuguese and second for numbers less than 31 within three whitespace separated tokens from each other. Thus, an example matching date pattern would be ”10 de enero”.
%
%
%        {\em Probability}: Posterior probability of tweet being Civil Unrest related Given User-type
%
%        {\em Event Geocoding}: If no location in Text then location of retweeter is used based on HRL's Geocoding methodology.
%
%        {\em EventType and Population}: Duplicate Forecasts for same date/location are combined into one and then tweet history of every tweeter/retweeter in the forecast are searched for keywords pertaining to particular classes identified by domain expert and then most commonly occuring class is assigned.
%        
%        {\em Results Provided}: Counts of Warnings provided and results of Manual examination of warning provided
%
%    \item \textbf{HRL's Tumblr Based Paper- Civil Unrest Prediction: A Tumblr-Based Exploration}
%
%        {\em keywords}: 59 keywords by domain experts.
%
%        {\em location}: Text Based filter for pre-defined locations(1022 in number)
%
%        {\em date}: Similar to the Twitter paper
%
%        {\em Dataset}: Full tumblr firehose of public post from 2013-04 to 2013-11. Evaluations from 2013-06 to 2013-08. Results presented in terms of Number Events Detected, Precision, Leadtime
%
%    \item \textbf{Identifying Content for Planned Events Across Social Media Sites - Luis Gravano-Columbia University}:
%            Not exactly related to ours. It deals with identifying user contributed content for future planned events that are already known like music concert etc.
%
%\end{itemize}

Three categories of related work -- \emph{Event Detection, Extraction of Planned Events and  Event Forecasting} -- are briefly discussed here.

First, Event Detection/Extraction from textual News has been studied extensively in literature. \cite{Allan:2002:TDT} \cite{Yang:1998:SRO}\cite{Gabrilovich:2004:NPP} make use of document clustering techniques to identify events retrospectively or as the stories arrive.\cite{Chambers:2011:TIE},\cite{Banko07openinformation}, \cite{Riloff:2003:LEP} talk about extraction patterns/templates to extract information from Text. \cite{Ritter:2012} shows it's possible to accurately extract a calendar of significant events from Twitter by training a tagger for recognizing event phrases.\cite{Sankaranarayanan:2009:TNT} captures tweet clusters of interest to identify late breaking News from twitter.In an altogether different application \cite{Sakaki:2010:EST} observes tweets to enable detection of occurences of Earth Quakes promptly.

Second, some work has been done regarding extraction of planned/future mentions of events from Social Media. Recorded Future\cite{recordedFuture} is an analytics company that performs real-time analysis of news and tweets to identify mentions of future events.\cite{tops2013predicting} and \cite{bosch2013estm} use classification and regression techniques to identify the time to an event referred to by a tweet.In \cite{tops2013predicting} a tweet was classified into one of the several identified equal length time bins. \cite{Jatowt:2011:ECE} tries to provide a collective image of the future associated with an entity summarizing all future related information available.In \cite{Becker:2012:ICP} and \cite{Becker_automaticidentification} content about known planned events is identified from Social Media. \iffalse \sathappanc{we do it on-line and focus mainly on planned protest and do it in multiple sources} \fi
%\cite{baeza2005searching},\cite{dias2011future},\cite{tops2013predicting},\cite{Jatowt:2011:ECE},\cite{Kawai:2010:CSE} talk about future retrieval or extraction of mentions of Future events.
Several work has been done on temporal information extraction from text.
\cite{Kawai:2010:CSE} presents a search engine ChronoSeeker for searching future and past events.It makes use of an SVM Classifier to disambiguate between the various temporal expressions in a document. .\cite{baeza2005searching} and \cite{dias2011future} also try to extract future temporal references from text, with the latter using a classifier approach to differentiate between a planned event and a rumor.\iffalse \sathappanc{Also cite TEMPEX, TIMEN etc} \fi
Third, few work has been done in the area of Event Forecasting. \cite{Radinsky:2013:MWP}, is one of the first papers in this category. The paper learns event sequences from a corpora spanning over 22 years and then uses these sequences to forecast events like disease outbreaks, deaths and riots.It only predicts if an event of interest will happen in the future given the sequence of events seen but it does not predict when/where(city level) that event will happen. In \cite{DBLP:journals/corr/Kallus14}, the author makes use of data from Recorded Future to find if a  significant protest event will occur in the next three days or not usig a random forest classifier.The author only focuses on prediction of significant events and also the forecast is limited to the next three days.\cite{compton2013detecting} and \cite{xu2014civil} are two pieces of research  that are very close to our line of work. Both papers follow similar methodologies but are based on different datasets--Twitter and Tumblr.In \cite{compton2013detecting}, a list of 335 keywords identified by experts is used to filter twitter stream and the filtered twitter streams are then searched for the presence of future dates in a naive manner by first searching for month names and then for a number less than 31. Such an approach, will not be capable of finding relative mentions of future dates like "tomorrow", "next tuesday" etc. If any location is mentioned in the text of the tweet it is used as the event location else the location is determined based on a in-house geocoder\cite{hrlgeocoder}.\cite{xu2014civil} works on Tumblr and makes uses of a fewer set of keywords(59) to filter the Tumblr feed. The filtered feed is further refined by searching for mentions of around 1022 different location names.Finally, the documents are searched for a future date in the same way as in \cite{compton2013detecting}.

\begin{table*}
    \centering
    \caption{comparison of our approach with other future event detection methods}
    \begin{tabular}{l c c c c c }
        \hline
            & Domain Specific & Multi-Source & Geo-Coding & Temporal Normalization & Feature 4 \\
        \hline
        ~\cite{Kawai:2010:CSE, bosch2013estm} & & & &\checkmark& \\
        ~\cite{xu2014civil} &\checkmark & & \checkmark& &\\
        reference set 3 & &\checkmark& & &\\
        our method &\checkmark &\checkmark &\checkmark &\checkmark &\checkmark\\ 
\end{tabular}
\end{table*}
