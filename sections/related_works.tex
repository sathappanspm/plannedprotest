Four categories of related work 
%-- \emph{Event Detection, Temporal Information Extraction, Future Retrieval, Extraction of Planned Events and Event Forecasting} -- 
are briefly discussed here.

\begin{table*}
    \centering
    \vspace{-2em}
    \caption{Comparison of our approach against other future retrieval techniques.}
    \begin{tabular}{l p{1.5cm} p{1.5cm} p{1.5cm} p{1.5cm} p{3cm}}%{|*{17}{c|}}
        \hline
        & Relative date resolution & Ingest multiple sources? & Reasoning about location & Learning word/phrase filters \\
        \hline
        `Future' Search Engines~\cite{Kawai:2010:CSE, Jatowt:2011:ECE,baeza2005searching}&\checkmark & & \\
        Time-to-Event Recognition~\cite{tops2013predicting, bosch2013estm}&\checkmark & & \\
        Planned Protest Detection~\cite{xu2014civil,compton2013detecting} & &\checkmark & &\\ 
        This paper &\checkmark &\checkmark &\checkmark&\checkmark\\  \hline
\end{tabular}
\label{comp-table}
\end{table*}

{\bf Event detection via text extractions}
is an extensively studied topic in the literature. Document clustering techniques are used 
in~\cite{Allan:2002:TDT, Yang:1998:SRO, Gabrilovich:2004:NPP} to identify events retrospectively or as the stories arrive.
Works like~\cite{Banko07openinformation, Chambers:2011:TIE, riloff2003learning} focus on
extraction patterns (templates) to extract information from text. Ritter et al.~\cite{Ritter:2012} show that
it is possible to accurately extract a calendar of significant events from Twitter by training a tagger for recognizing event phrases.
\iffalse 
Sankaranarayanan et al.~\cite{Sankaranarayanan:2009:TNT} captures tweet clusters of interest to identify late breaking News from twitter 
\fi.
Highly specialized applications
also exist; e.g., Sakaki et al.~\cite{Sakaki:2010:EST} mine tweets to enable prompt detection of occurences of earthquakes.

{\bf Temporal information extraction} is another well studied topic.
The TempEval challenge~\cite{tempeval} led to a significant amount of
algorithmic development for temporal NLP.
For instance, a specification lanuage
for temporal and event expressions in natural language text is described in~\cite{timeml}.
Refs.~\cite{LlorensDGS12} and~\cite{tempex} provide methods to resolve temporal expressions in text (our own
work here uses the TIMEN package~\cite{LlorensDGS12}).

{\bf Event forecasting} is a burgeoning area. 
Radinsky and Horvitz~\cite{Radinsky:2013:MWP} find event sequences from a corpora and then use these sequences to determine if 
an event of interest (e.g., a disease outbreak, or a riot)
will occur sometime in the future. This work predicts only if a potential event will happen given a historical event sequence
but does not geolocate the event to a city-level resolution, as we do here.
Kallus~\cite{nathankallus} makes use of event data from 
RecordedFuture~\cite{recordedFuture} to determine if a  significant protest event will occur in 
the subsequent three days and casts this as a classification problem.
This work only focuses on prediction of significant events (suitably defined) and
the forecast is limited to the next three days. Ramakrishnan et al.~\cite{emberskdd} describe the EMBERS
system for forecasting civil unrest using open source indicators but this work is primarily focused on shallow mining of
a broad set of data sourceas in contrast to the focused analysis of planned protest announcements that we study here.

Finally, the area of research most closely related to our work is the emerging topic of
{\bf future retrieval}.
Baeza-Yates~\cite{baeza2005searching} providing one of the earliest discussions
of this topic; here future temporal information in text is found and used to retrieve content from search queries that 
combine both text and time with a simple ranking scheme. Kawai et al.~\cite{Kawai:2010:CSE} present a search engine (ChronoSeeker) for searching 
future and past events.
They make use of an SVM classifier to disambiguate between the various temporal expressions in a document.
Dias et al.~\cite{dias2011future} classify web snippets into three classes depending on if a future date can/cannot be predicted 
from the snippet or if it is a rumor.
%{\bf Extraction of planned or future mentions of events from social media} is a very popular topic in social media
%analytics.
RecordedFuture~\cite{recordedFuture}, introduced earlier, conducts
real-time analysis of news and tweets to identify mentions of events along with associated times. Anectodally it 
is estimated that approximately (only) 5--7\% of events extracted 
by RecordedFuture are about the future.
Tops et al.~\cite{tops2013predicting} aim to classify a tweet talking about an event into discrete time segments and thereby predict the 
`time to event'.
Bosch et al.~\cite{bosch2013estm} use regression techniques to identify the time to an event referred to by a tweet.
Jatowt et al.~\cite{Jatowt:2011:ECE} provide a collective image of the future associated with an entity summarizing all future related information available.
Becker et al.~\cite{Becker:2012:ICP} try to identify more content about known planned events (e.g., a concert) across social media. This work
for instance assumes that we know the event beforehand and aims to identify relevant details of the event.

%\cite{baeza2005searching},\cite{dias2011future},\cite{tops2013predicting},\cite{Jatowt:2011:ECE},\cite{Kawai:2010:CSE} talk about future retrieval or extraction of mentions of Future events.
In particular, two publications---Compton et al. ~\cite{compton2013detecting} and Xu et al.~\cite{xu2014civil}---align very closely to our own work as their emphasis is on protest forecasting.
Both works are aimed at forecasting protests
but emphasize different data sources and different methodologies. For instance, the work in~\cite{compton2013detecting} filters the Twitter stream for
keywords of interest and searches for future date mentions in only absolute terms, i.e., explicit mentions of a month name and a number (date)
less than 31. 
Such an approach will not be capable of extracting the more
common way in which future dates are referenced, e.g., phrases like
``tomorrow,'' ``next tuesday.'' 
%Any location mentioned in the tweet text is used as the event location. If there are no location mentions then the location is determined based on~\cite{hrlgeocoder}. 
The work in~\cite{xu2014civil} by the same group of authors uses the Tumblr feed with a smaller set of keywords but
again is restricted to the use of absolute time identifiers.

In surveying the state-of-the-art, we arrived at desiderata for a planned protest forecasting system. As shown in Table~\ref{comp-table},
we desire a system that is capable of: i) ingesting a broad range of data sources from both popular news and social media,
ii) learning relevant phrases for tracking protests, 
iii) handling relative mentions of dates, and
iv) providing a rich representational and reasoning basis for location. 
As Table~\ref{comp-table} summarizes, current systems provide only partial solutions and the proposed
approach addresses all four desired criteria.
